\documentclass{article}

% Listings
\usepackage{listings}
\usepackage{color}
\usepackage{upquote} % preserve backticks

% Fonts
%\usepackage{palatino}
\usepackage{inconsolata}
\usepackage[T1]{fontenc}

% Margin
\usepackage[margin=0.5in]{geometry}

% Images
\usepackage{graphicx}
\graphicspath{ {images/} }


% Colors
\definecolor{comment}{rgb}{0.31,0.31,0.31}
\definecolor{codegray}{rgb}{0.19,0.19,0.19}
\definecolor{string}{rgb}{0.56,0.66,0.35}
\definecolor{backcolor}{rgb}{0.96,0.96,0.96}
\definecolor{keyword}{rgb}{0.67,0.25,0.25}
 
\lstdefinestyle{mystyle}{
    backgroundcolor=\color{backcolor},
    commentstyle=\color{comment},
    keywordstyle=\color{keyword},
    numberstyle=\tiny\color{codegray},
    stringstyle=\color{string},
    basicstyle=\footnotesize\ttfamily,
    breakatwhitespace=false,
    breaklines=true,
    captionpos=t,
    keepspaces=true,
    numbers=left,
    numbersep=5pt,
    showspaces=false,
    showstringspaces=false,
    showtabs=false,
    tabsize=2
}

\lstset{style=mystyle}
\lstMakeShortInline[columns=fixed]|



\title{Homework 2 - System Monitoring and VM Networking}
\author{Jared Klingenberger}
\date{February 27, 2016}

\begin{document}

\maketitle

\lstlistoflistings

\section{Kessler Paper}

\subsection{ARP and Types of Addressing}

The Address Resolution Protocol (ARP) is a method for determining the MAC
address correlated with an IP address it needs.

A machine address differs from a network address in that it is used on a lower
logical layer of the network stack. MAC addresses are used to refer to a
specific interface on a specific machine, whereas a network address like an IP
address is used for the more general concept of finding and addressing
machines between different networks.

\subsection{DHCP}

The Dynamic Host Configuration Protocol (DHCP) was created to provide a way
to assign IP addresses to host on a network in an automated fashion. This is
useful for two reasons:

\begin{itemize}
    \item Users don't have to manually choose an IP address.
    \item Administrators don't have to keep IP addresses reserved for longer
          than necessary.
\end{itemize}

To make things even simpler, DHCP is useful for networks that change Internet
Service Providers and subsequently changing their IP address. Without DHCP,
admins would need to reassign IP addresses on all their machines in this
scenario.

At a high level, DHCP works by the following steps:

\begin{enumerate}
    \item Send a broadcast message asking for an available DHCP server.
    \item Receive responses from candidate DHCP server(s).
    \item Choose a DHCP server and ask it for an IP address.
    \item DHCP server responds with an IP address that is good until end of
          lease.
    \item When the lease ends, ask for a renewal.
    \item Gracefully end the DHCP lease.
\end{enumerate}

\section{System Monitoring}

I created the script discussed below and named it |hw2-Q2.sh|. Then, in order
for cron to use it, I issued a |`sudo crontab -e`| to edit the crontab. Then,
I placed the following line to get the program to run once a day at 5am:

\begin{lstlisting}
0 5  *   *   *     cd /home/jared/Documents/code/udpecho; ./hw2-Q2.sh > /tmp/cron-hw2-$(date +\%F).log 2>&1
\end{lstlisting}

\subsection{Config file}

The config file in Listing \ref{code:conf} is read by the monitoring script in
Listing \ref{code:q2}. To make things simple, the config file is simply
|source|d by the script as if it were a bash file. Thus, the format for
setting variables is the same as one would do in any bash script.

\subsubsection{Variables}

\begin{itemize}
    \item |disk_max| (percent) - disk usage threshold.
    \item |bw_max| (kbits/s) - total bandwidth usage threshold per interface.
\end{itemize}

\subsection{Disk Usage}

This part of the script simply calls the UNIX program |du| and picks out the
disk usage info for the root filesystem. If the percentage threshold specified
in the config file is exceeded, the script will calculate and print the disk
usage by each user's folder in the |/home| directory. Values are reported by
percentage.

\subsection{Authentication Failure Monitoring}

Here, I decided to go a different route than parsing through
|/var/log/auth.log|. My motivation for doing this is that there seems to be no
standard searchable message for different authentication failures. For
example, failing to authenticate to the root user when using |su| results in
this log message:

\begin{lstlisting}
Feb 27 21:42:56 hotdog su[4642]: pam_unix(su:auth): authentication failure; logname=jared uid=1000 euid=0 tty=/dev/pts/10 ruser=jared rho
st=  user=root
Feb 27 21:42:58 hotdog su[4642]: pam_authenticate: Authentication failure
Feb 27 21:42:58 hotdog su[4642]: FAILED su for root by jared
Feb 27 21:42:58 hotdog su[4642]: - /dev/pts/10 jared:root
\end{lstlisting}

While failing to authenticate using |sudo| results in a different message:

\begin{lstlisting}
Feb 28 01:50:40 hotdog sudo:    jared : 1 incorrect password attempt ; TTY=pts/1 ; PWD=/home/jared/Documents/code/udpecho ; USER=root ; COMMAND=/bin/echo hi
\end{lstlisting}

So, I chose to install the program |auditd| and use one of its utility
programs |aureport| to get more accurate authentication statistics. This
program is available in the Ubuntu repositories. Some of its better features
that I made use of were its ability to report just the authentication failures
and its ability to filter only events within particular time ranges. In this
case, the script outputs the number of failures over the past 24 hours.

\subsection{Bandwidth Measurement}

This was a more challenging part of the project mainly because of drawbacks
with the particular network monitoring tool I chose to use, |vnstat|. This
program is similarly available in the Ubuntu repositories. While it's a great
program to use on the command line for getting a quick idea of what's going on
with the network, it was more difficult to script against without mucking with
its database.

Ultimately, I had the program list out the available |eth*| interfaces, then I
measure the bandwidth usage on each interface for 5 seconds. Since |vnstat|
only outputs separate send and receive measurements, I had to do some manual
adjustments to add these two numbers together with correct units in order to
get a picture of total bandwidth usage.

\subsection{Sample output}

\begin{lstlisting}
Disk usage on /: 60%

Failed login attempts by user:
------------------------------
8 root
7 jared

Analyzing bandwidth usage by interface:
---------------------------------------

Bandwidth used over 5 seconds on interface eth0...
24.00 kbits/s

Bandwidth used over 5 seconds on interface eth1...
20130.00 kbits/s
WARNING: bandwidth threshold 100 kbits/s exceeded!
\end{lstlisting}

\section{VM Networking}

The script shown in Listing \ref{code:q3} is a pretty straightforward
implementation of the problem statement in the assignment document. A
parameter valued 1 through 4 is provided to the script to identify which part
of the question the script should perform. The first two parts simply run
|iperf3| for 20 seconds, whereas the second pair of parts run |iperf3| while
simultaneously running |UDPEchoClient| at varying rates.

Program output for parts 1 through 4 can be found in Listing \ref{log:q3p1}
through Listing \ref{log:q3p4server}.

\subsection{Discussion}

The results of the first two parts (UDP and TCP |iperf3| executions) can
largely be seen as controls for the experiments I performed in parts 3 and 4.
The UDP execution saw a packet drop rate of roughly 0.25 percent, while the
TCP execution window size had a mode of 130 KBytes.

In parts 3 and 4, I varied the interval delay by values $10^{-i}$, where $i
\in \{1,2,3,4\}$. Most of the results in part 3 were unsurprising: in order of
decreasing delay time, I saw packet losses of 1.6, 0.2, 1.2, and 2.4 percent.
Most of these seemed to make sense, except for the 0.2 percent result
attributed to the delay time of 0.1 s. Jitter was another interesting
variable; in order of decreasing delay times I saw 0.099, 0.051, 0.456, and
0.114 ms jitter. Again, the 0.1 s delay time seemed to outperform the slower
delay time of 1 s.

The results of part 4 were similarly difficult to predict. Some delay values
resulted in large window sizes, while others led to smaller ones. There did
not seem to be a strong correlation in any case.

\section{Code}

\lstinputlisting[language=bash,caption=hw2-Q2.sh,frame=single,label=code:q2]{hw2-Q2.sh}

\lstinputlisting[language=bash,caption=monitor.conf,frame=single,label=code:conf]{monitor.conf}

\lstinputlisting[language=bash,caption=hw2-Q3.sh,frame=single,label=code:q3]{hw2-Q3.sh}

\lstinputlisting[caption=q3p1.log,frame=single,label=log:q3p1]{logs/q3p1.log}
\lstinputlisting[caption=q3p1-server.log,frame=single,label=log:q3p1server]{logs/q3p1-server.log}
\lstinputlisting[caption=q3p2.log,frame=single,label=log:q3p2]{logs/q3p2.log}
\lstinputlisting[caption=q3p2-server.log,frame=single,label=log:q3p2server]{logs/q3p2-server.log}
\lstinputlisting[caption=q3p3.log,frame=single,label=log:q3p3]{logs/q3p3.log}
\lstinputlisting[caption=q3p3-server.log,frame=single,label=log:q3p3server]{logs/q3p3-server.log}
\lstinputlisting[caption=q3p4.log,frame=single,label=log:q3p4]{logs/q3p4.log}
\lstinputlisting[caption=q3p4-server.log,frame=single,label=log:q3p4server]{logs/q3p4-server.log}

\end{document}
